% LaTeX file for resume
% This file uses the resume document class (res.cls)

\documentclass[margin]{res}
\usepackage [brazil]{babel}      % nomes e hifenaçã em português

\usepackage{t1enc}               % Permite digitar os acentos de forma normal
\usepackage[utf8]{inputenc}

\topmargin=-0.5in                % start text higher on the page
\setlength{\textheight}{10in}    % increase text height to fit resume on 1 page

\begin{document}

\name{\textit{Guilherme Silveira dos Santos}}
\address{Berlin, DE \\ xguiga@gmail.com \\ Phone: +49 (173) 979-9383 }

\begin{resume}

\section{Summary}
I’m a Computer Engineer since 2011, but I started programming long time before that. I consider myself autodidact because I like to learn new technologies all the time and I can do it very quickly. I have started to study PHP at home since I was teenager, what give me good knowledge to develop many web applications. During the college and internships I learned other programming languages, such as C, C++, Java and Lua and I could improve myself with other skills.

I’ve been working for long time as a freelancer developer and most of the time involving web development, as a result I became an experienced web developer with big expertise in PHP, JavaScript, CSS and HTML. In 2015 I started working with Node.js and Golang.

I’ve already contributed in many open source projects like Enlightenment Foundation Libraries (EFL), Zend Framework and other small JavaScript and Golang libraries. I like to work with open source projects and I do it in my spare time.

I worked in companies with several different technologies and it made me an open minded developer, with can-do attitude and a good interpersonal relationship. Currently I’m working mainly with PHP and Golang as Backend Developer in Berlin, but looking for new challenges around the world.

\section{Education} UNIVALI University, BSc in Computer Engineering, June 2011.

\section{Experience}
\vspace{-0.1in}
    \begin{tabbing}
    \hspace{2.3in}\= \hspace{1.7in}\= \kill % set up two tab positions
    \textbf{Ridelink}    \>\>\textbf{Sep 2016 - Present}\\
    \textit{Backend Developer}\\
    \textbf{Main Technologies}: PHP, Symfony, Silex, Golang, Python, MySQL, ElasticSearch, \\Docker, AWS;
    \end{tabbing}\vspace{-20pt}      % suppress blank line after tabbing
    \vspace{2mm}
Development of Car Sharing platform. We're working with microservice over AWS using SQS, ECS, ElasticCache, ElasticSearch, RDS, CloudWatch, etc. Our microservices are developed in Golang, PHP using Symfony or Silex and we also have some projects in Python. We're improving our RESTful API and adding new features.

\vspace{-0.1in}
    \begin{tabbing}
    \hspace{2.3in}\= \hspace{1.7in}\= \kill % set up two tab positions
    \textbf{Neoway Business Solution}    \>\>\textbf{Jun 2015 - Sep 2016}\\
    \textit{Backend Developer}\\
    \textbf{Main Technologies}: Golang, ElasticSearch, MongoDB, RabbitMQ, Docker, Rkt,\\CoreOS, AWS;
    \end{tabbing}\vspace{-20pt}      % suppress blank line after tabbing
    \vspace{2mm}
Development of Big Data platform to Market Intelligence. We're using mainly Golang but Node.js is also used in a couple of projects. I develop several RESTful API to communicate with different back-ends like: ElasticSearch, MongoDB and RabbitMQ what give me some experience how to use and configure them.

We love dev-ops culture here, for that I've developed some tools to automatize our deploy at AWS - Amazon Web Services. We often write unit and integration tests to make our deploy as continuous as possible using gitlab flow and docker/rkt containers.

\vspace{-0.1in}
    \begin{tabbing}
    \hspace{2.3in}\= \hspace{1.7in}\= \kill
    \textbf{uTech Tecnologia}    \>\>\textbf{Nov 2014 - Feb 2015}\\
    \textit{Full Stack Developer}\\
    \textbf{Main Technologies}: C++, Qt, QML, JavaScript, SIP;
    \end{tabbing}\vspace{-20pt}
    \vspace{2mm}
Outsourced development of a Softphone to integrate with company’s platform. The biggest challenge was be multi-platform, running in Windows 7, Linux and Mac OS X. The software was developed using C++, QML and JavaScript through Qt library and PJSIP as SIP stack.    

\vspace{-0.1in}
    \begin{tabbing}
    \hspace{2.3in}\= \hspace{1.7in}\= \kill
    \textbf{GIOX Tecnologia}    \>\>\textbf{Mar 2013 - Present}\\
    \textit{CEO and Full Stack Developer}\\
    \textbf{Main Technologies}: PHP, JavaScript, HTML5, CSS3, MySQL, Phing, Ansible;
    \end{tabbing}\vspace{-20pt}
    \vspace{2mm}
Development of ERP to small business with electronic invoice (NF-e - Nota Fiscal Eletrônica). To develop this project I used PHP, MySQL, JavaScript, HTML5 and CSS3. The project was based in Zend Framework 2, PHPUnit and Doctrine2 in the back-end. It communicated with Web Services using SOAP protocol and XML. In the front-end I used Bootstrap3, jQuery, Underscore.js and Backbone.js.

I've improved myself as system administrator and dev-ops because I needed to create all the infrastructure used to deploy and run the application, for that I used Digital Ocean platform, and I needed to install, configure and maintain some Linux servers, I could automate a lot of tasks using Ansible, Git and Phing.

\vspace{-0.1in}
    \begin{tabbing}
    \hspace{2.3in}\= \hspace{1.7in}\= \kill
    \textbf{Digitro Tecnologia}    \>\>\textbf{Dec 2009 - Mar 2013}\\
    \textit{Backend Developer}\\
    \textbf{Main Technologies}: C, C++, Lua, SIP, uCLinux, GStreamer, ShellScript, Blackfin;
    \end{tabbing}\vspace{-20pt}
    \vspace{2mm}
Embedded development of an IP Phone touch screen with color display using Blackfin processor with uCLinux distribution. We used u-boot, EFL graphic library, GLib, GObject, GDBus, CppUTest and Sofia-SIP as SIP stack.
    
Speaker recognition: Web service responsible for creating audio models from the voice and storing it. The voice models were used later to identify someone talking on a audio recording. Tools used: Lua, lighthttpd, GStreamer, fastcgi, MongoDB.

Keyword spotting: Middleware used to perform text search on audio using a proprietary protocol to communicate with clients. Tools used: Lua, C, GStreamer.

Flash Audio streaming server: Web service responsible for loading audio recordings in different audio codecs, transcode then, apply filters and effects and send to the client through the RTMP protocol. Tools used: C++, C, GStreamer, Monit.

\section{Skills Base} \textit{Programming Languages}: Golang, JavaScript, PHP, Lua, C, C++ and a little of Java;

	\textit{Databases}: MongoDB, MySQL, LevelDB and a little of PostgreSQL;
	
	\textit{JavaScript Frameworks}: jQuery, Backbone.js, Underscore.js, Angular.js;

	\textit{Agile practices}: TDD, SCRUM, Kanban, Pair programming, Clean code, Code Review;
	
	\textit{Tools}: Git, Makefile, Phing, Grunt, Bower, Ansible, Vagrant, Jenkins;
	
	\textit{Languages}: Fluent in Portuguese, Intermediate in English and Spanish;
	
	\textit{Others}: ElasticSearch, RabbitMQ, Docker, Rkt;

\section{More Info} \textit{Linkedin}: https://www.linkedin.com/in/guilhermesilveirasantos

    \textit{Github}: https://github.com/guilherme-santos

\end{resume}
\end{document}
