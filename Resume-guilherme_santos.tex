% LaTeX file for resume 
% This file uses the resume document class (res.cls)

\documentclass[margin]{res}
\usepackage [brazil]{babel}     % nomes e hifenaçã em português
 
\usepackage{t1enc}              % Permite digitar os acentos de forma normal
\usepackage[utf8]{inputenc} 

\topmargin=-0.5in  % start text higher on the page
\setlength{\textheight}{10in} % increase text height to fit resume on 1 page
\begin{document}  
\name{\textit{Guilherme Silveira dos Santos}}

\address{Florianopolis, BR \\ guilherme@giox.com.br \\ Phone: +55 (48) 9640-3883 }
                           
                        
\begin{resume}                        
 
\section{Summary}   
I am a Computer Engineer since 2011, but I started programming long time before that. I have always been learning new things and I can do it very quickly. I started studying PHP at home when I was 13, since that, I never stopped using it. During the college and internships I learned other programming languages, such as C/C++, Java and Lua.

I’ve been working long time as a freelancer developer and most of the time involving web development, as a result I became an experienced web developer with big expertise in PHP, JavaScript, CSS and HTML.

In my spare time I’ve contributed in many open source projects like Enlightenment Foundation Libraries (EFL), Zend Framework and Backbone.js.

I worked in many companies with several technologies and it made me an open minded developer, with can-do attitude and a good interpersonal relationship. Currently I’m working as Full Stack Developer in south Brazil, but looking for new challenges around the world.
 
\section{Education} UNIVALI University, BSc in Computer Engineering, June 2011.
  
\section{Experience}

\vspace{-0.1in}
   \begin{tabbing}
   \hspace{2.3in}\= \hspace{1.7in}\= \kill % set up two tab positions
    \textbf{GIOX Tecnologia}    \>\>\textbf{Apr 2013 - Present}\\
    \textit{CEO and Full Stack Developer}\\        
    \textbf{Main Technologies}: PHP, JavaScript, HTML5, CSS3, Linux, Git;
   \end{tabbing}\vspace{-20pt}      % suppress blank line after tabbing
    \vspace{2mm}
I'm the main developer of an ERP (Enterprise Resource Planning) system created by this company.
The backend was developed using PHP and MySQL, the frontend using HTML5, CSS3 and JavaScript. Communication with some Web 
Services. Development of REST API. Some libraries used: Zend Framework 2, PHPUnit, Doctrine2, Boostrap3, jQuery and Backbone.js.

Installation, configuration and maintenance of Linux servers to run the application, backup, and database using Ansible, Git and Phing to automate the process.
\vspace{-0.1in}
   \begin{tabbing}
   \hspace{2.3in}\= \hspace{1.7in}\= \kill % set up two tab positions
    \textbf{uTech Tecnologia}    \>\>\textbf{Nov 2014 - Feb 2015}\\
    \textit{Outsourced Developer}\\        
    \textbf{Main Technologies}: C++/QT/QML, JavaScript, SIP;
   \end{tabbing}\vspace{-20pt}      % suppress blank line after tabbing
    \vspace{2mm}
This was an outsourced project to develop a Softphone. It became part of company’s 
platform. The biggest challenge was be multi-platform, running in Windows 7, Linux and Mac OS X.
The software was developed using C++, QML and JavaScript through Qt library and PJSIP as SIP stack.

   \begin{tabbing}
   \hspace{2.3in}\= \hspace{1.7in}\= \kill % set up two tab positions
    \textbf{Digitro Tecnologia}    \>\>\textbf{Dec 2009 - Mar 2013}\\
    \textit{Software Engineer}\\   
    \textbf{Main Technologies}: C/C++, Lua, Shell Script, Linux, SIP;
   \end{tabbing}\vspace{-20pt}      % suppress blank line after tabbing
    \vspace{2mm}

Embedded development of an IP Phone touch screen with color display 
using Blackfin processor with uCLinux distribution. We used u-boot, EFL 
graphic library, GLib, GObject, GDBus, CppUTest and Sofia-SIP as SIP stack.

Speaker recognition: Web service responsible for creating audio models from the voice and storing it. The voice models where used later to identify someone talking on a audio recording. Tools used: Lua, lighthttpd, GStreamer, fastcgi, MongoDB.

Keyword spotting: Middleware used to perform text search on audio using a proprietary protocol to communicate with clients. Tools used: Lua, C, GStreamer.

Flash Audio streaming server: Web service responsible for loading audio recordings in different audio codecs, transcode then, apply filters and effects and send to the client through the RTMP protocol. Tools used: C++, C, GStreamer, Monit.

\section{Skills Base} \textit{Progamming Languages}: PHP, JavaScript, C/C++, Lua, Java;

	\textit{Databases}: MySQL, MongoDB, plus some experience with Postgres;
	
	\textit{Agile practices}: TDD, SCRUM, Kanban, Pair programming, Clean code;
	
	\textit{Languages}: Fluent in Portuguese, Intermediate in English and Spanish;
	
	\textit{Tools}: Phing, Grunt, Ansible, Vagrant, Docker, Jenkins;
 
\section{More Info} \textbf{Linkedin}: https://www.linkedin.com/in/guilhermesilveirasantos/en

    \textbf{Github}: https://github.com/guilherme-santos

\end{resume} 
\end{document}









